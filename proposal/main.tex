\documentclass[a4paper,10pt]{article}

%page geometry
\usepackage[colorlinks,linkcolor=blue,citecolor=blue,urlcolor=blue]{hyperref}
\usepackage[lmargin=2cm,rmargin=2cm,tmargin=1.25cm,bmargin=1.25cm,includefoot,includehead]{geometry}

%typography
\usepackage[T1]{fontenc}
\usepackage[utf8]{inputenc}
%% Serif Times fonts
\renewcommand{\rmdefault}{ptm} 
%% Sans-serif Arial-like fonts (Helvetica)
\renewcommand{\sfdefault}{phv} 

%text structures
\usepackage{enumitem}
\usepackage{tabu}
\usepackage[dvipsnames]{xcolor}
\usepackage{tcolorbox}
\usepackage{colortbl}
\usepackage{graphicx}
\usepackage{eurosym}
\usepackage{xspace}

\usepackage{tabularx}
\usepackage{boxedminipage}
\usepackage{spreadtab}



%uncomment if you want to draw a Gantt diagram
%\usepackage{pgfgantt}



%couleurs ARN 
\definecolor{ANRblue}{HTML}{004d95}
\definecolor{ANRred}{HTML}{FF0000}
\definecolor{ANRpurple}{HTML}{800080}


%physique-chimie
\usepackage{siunitx} %unités en \SI{10}{\micro\metre}
\usepackage[version=3]{mhchem} % Formula subscripts using \ce{}

\usepackage{csquotes}

%commandes pour dire qui doit faire quoi
\newcommand{\Mathieu}[1]{\textcolor{orange}{#1}}
\newcommand{\Helene}[1]{\textcolor{purple}{#1}}

% %biblio
%\usepackage[backend=biber,style=numeric-comp, language=british,eprint=false, url=false, doi=false, sortcites=true, sorting=none, isbn=false, firstinits=true,maxbibnames=99]{biblatex}
%\renewbibmacro{in:}{%
%  \ifentrytype{article}{}{%
%  \printtext{\bibstring{in}\intitlepunct}}}
%\bibliography{pre-propal-2017}
%\bibliographystyle{plain}

%no month
%\AtEveryBibitem{\clearfield{month}}

%\addbibresource{IEEEabrv.bib}

%lien DOI ou URL si disponible
%\newbibmacro{string+doi}[1]{%
%  \iffieldundef{doi}{%
%    \iffieldundef{url}{#1}{\href{\thefield{url}}{#1}}}{\href{http://dx.doi.org/%\thefield{doi}}{#1}}}

%sur le titre, en couleur
%\DeclareFieldFormat
%  [article,inbook,incollection,inproceedings,patent,thesis,unpublished]
%  {title}{{\textcolor{ANRblue}{#1\addperiod}}}
  

%Our names in bold
\usepackage{xstring}
\usepackage{etoolbox}
%\newboolean{bold}
%% \newcommand{\makeauthorsbold}[1]{%
%% \DeclareNameFormat{author}{%
%%     \setboolean{bold}{false}%
%%     \renewcommand{\do}[1]{\ifstrequal{##1}{####1}{\setboolean{bold}{true}}{}}%
%%     \docsvlist{#1}%
%%     \ifnumequal{\value{listcount}}{1}%
%%     {\ifnumequal{\value{liststop}}{1}%Single author
%%       {\expandafter\ifthenelse{\boolean{bold}}{\textcolor{ANRpurple}{\underline{##1\addcomma\addspace ##4\addcomma\isdot}}}{##1\addcomma\addspace ##4\addcomma\isdot}}
%%       %first author
%%       {\expandafter\ifthenelse{\boolean{bold}}{\textcolor{ANRpurple}{\underline{##1\addcomma\addspace ##4}}}{##1\addcomma\addspace ##4}}}
%%       %last author
%%     {\ifnumless{\value{listcount}}{\value{liststop}}
%%       {\expandafter\ifthenelse{\boolean{bold}}{\addcomma\addspace \textcolor{ANRpurple}{\underline{##1\addcomma\addspace ##4}}}{\addcomma\addspace ##1\addcomma\addspace ##4}}
%%       %middle author
%%           {\expandafter\ifthenelse{\boolean{bold}}{\addcomma\addspace \textcolor{ANRpurple}{\underline{##1\addcomma\addspace ##4\addcomma\isdot}}}{\addcomma\addspace ##1\addcomma\addspace ##4\addcomma\isdot}}%
%%     }%
%% }%
%% }
%% \makeauthorsbold{Cottin-Bizonne,Delanoë-Ayari,Leocmach}

%% %%hack fullcite so that it prints all authors
%% \makeatletter
%% \DeclareCiteCommand{\fullcite}
%%   {\defcounter{maxnames}{\blx@maxbibnames}%
%%     \usebibmacro{prenote}}
%%   {\usedriver
%%      {\DeclareNameAlias{sortname}{default}}
%%      {\thefield{entrytype}}}
%%   {\multicitedelim}
%%   {\usebibmacro{postnote}}
%% \makeatother

\newcolumntype{P}[1]{>{\raggedright}p{#1}}

%% section title format, in the text and in the table of content
\usepackage{tocloft}

\renewcommand{\cfttoctitlefont}{\color{ANRblue}\normalfont\scshape\bfseries\sffamily\Large}

\usepackage{titlesec}
\titleformat{\section}
{\color{ANRpurple}\normalfont\scshape\bfseries\sffamily\Large}
{\color{ANRpurple}\thesection}{1em}{}
\renewcommand{\cftsecfont}{\color{ANRpurple}\normalfont\scshape\bfseries\sffamily\large}

\titleformat{\subsection}
{\color{ANRblue}\normalfont\scshape\bfseries\sffamily\large}
{\color{ANRblue}\thesubsection}{1em}{}
\renewcommand{\cftsubsecfont}
{\color{ANRblue}\normalfont\scshape\bfseries\sffamily}

\titleformat{\subsubsection}
{\color{ANRpurple}\normalfont\bfseries\sffamily}
{\color{ANRpurple}\thesubsubsection}{1em}{}
\renewcommand{\cftsubsubsecfont}
{\color{ANRpurple}\normalfont\bfseries\sffamily\small}

\titleformat{\paragraph}[runin]
{\color{ANRblue}\normalfont\sffamily}
{\color{ANRblue}\theparagraph}{1em}{}

%titre, acronyme, défi, instrument
\newcommand{\myacro}[0]{XXXX}
\newcommand{\mytitle}[0]{\myacro: \\The long title of the ANR project}
\newcommand{\projectname}[0]{XXXX}
\newcommand{\mydomain}[0]{8.6}
\newcommand{\myinstrument}[0]{PRC}
\author{Coordinateur: Romain Michon\\
\small GRAME.
}

%% Headers
\usepackage{fancyhdr}
\pagestyle{fancy}
%\fancyhfoffset[re]{4cm}
\lhead{\myacro\\ Domaine \mydomain{} - \myinstrument}
\rhead{\LARGE\textcolor{ANRblue}{AAPG ANR 2019}}
\renewcommand{\headrulewidth}{0pt}

\lfoot{}
\cfoot{}
\rfoot{\thepage}%/\pageref{LastPage}} 




\title{\vspace{-\baselineskip}\mytitle}
\date{Domaine : Tansversal (?)\\
YYY}

\newcommand{\content}[1]{\emph{#1}\\} 


\begin{document}
\maketitle
\thispagestyle{fancy}



%\section{Executive summary }
\begin{tabular}{p{2.3cm} p{12cm}}
  \hline
  \textbf{Keywords} &   \\\hline
  \textbf{Challenge and Axe (?? A REVOIR)} & 
  \\\hline
\end{tabular}


%\section{Contexte, positionnement et objectif(s) de la proposition}
\section{Context, position and objectives of the proposal}
%Sur une page et demie (1,5) environ, 
%Décrire les objectifs et les hypothèses scientifiques
%Montrer l’originalité et la pertinence par rapport à l’état de l’art
%Décrire la méthodologie et la gestion des risques

\begin{figure}[h]
  \centerline{Il faut mettre des figure, ça fixe les idées}
    \caption{Il faut mettre des figure, ça fixe les idées (sagesse populaire)}
\label{fig1}
\end{figure}
%\section{Organisation du projet et moyens mis en œuvre}
\section{Project organization and implementation}

\paragraph{Human resources}
\nocite{*}
\paragraph{Project organization}
\paragraph{Budget (j'ai laissé le budget pour voir tableau avec spreadtab}

\begin{center}\small
  \begin{spreadtab}{{tabular}{|c|c|c|c|c|c|c|c|c||c|}}
\hline
 &@Insa & \multicolumn{3}{|c|}{@Centrale}  & @Qinteq & @eVaderis  & @Hager & @EDF &  @Total \\ 
\cline{3-5} & @Citi & @Ampere & @INL & @LTDS & & & & &\\ 
 & @(rate 100\%) & @(100\%) & @(100\%) & @(100\%) &   @(rate 35\%)&  @(rate 35\%)  &@(rate 30\%) &@(rate 30\%) &  @(k\euro)\\ \hline \hline
@Man Month ANR & 42 & 20& 20 &20  & 3 & 6 & 12 & 9 & \\ \hline
@Staff (k\euro)   & 148 & 78& 78 & 78 & 17 & 34 & 44 & 45 & sum(b5:i5) \\ \hline
@Equipment  (k\euro)   &  20 & 10 & 5 & 5  & & & & &sum(b6:i6) \\  \hline
@Travel  (k\euro)    &  20 & 5 & 5 & 5 &5 & 5 & 5 & 5 & sum(b7:i7)\\  \hline
@Management  (k\euro)    &  15 & 5 &5 & 5  & 5 & 5 & 5 & 5 & sum(b8:i8) \\
\hline\hline
@Total requested (k\euro)     & sum(b5:b8)  & sum(c5:c8)    & sum(d5:d8) & sum(e5:e8)& sum(f5:f8) & sum(g5:g8)& sum(h5:h8)& sum(i5:i8)& sum(j5:j8)  \\ \hline
\end{spreadtab}

\end{center}






%\section{Impact et retombées du projet}
\section{Project impact and benefits}

\bibliographystyle{plain}
\bibliography{main}

% \newpage
% \section{A enlever avant de soumettre: Taux de précarité (pas vérifié la formule 2018)}
% {Man month repartition per  partner }

% \begin{spreadtab}{{tabular}{|c||p{2.4cm}|p{2cm}|p{2.7cm}|p{2.5cm}||p{1cm}|}}
% \hline
% @Institution &
% @permanent
%  (institution) &
% @permanent 
%  (ANR) &
% @Non-permanent
%  (ANR) &
% @PhD thesis and internship &
% @Total
%  \\ \hline \hline
% @Insa Citi 	& 36 	& 	&  24	& 18 & sum(b2:e2) \\
% \hline
% @Centrale  	&  28	& 	&  24	&   36 & sum(b3:e3) \\
% \hline
% @Qinteq 	&  18	& 6  	&	& & sum(b4:e4) \\
% \hline
% @eVaderis 	&  18	& 6  	&	& & sum(b5:e5) \\
% \hline
% @Hager	&  18	& 6  	&	& & sum(b6:e6) \\
% \hline \hline 
% @Total 	&  sum(b2:b6)	&  sum(c2:c6)	& sum(d2:d6) & sum(e2:e6)  &  sum(b7:e7)\\
% \hline
% \multicolumn{6}{|l|}{\em taux de précarité (<<d7>>/(<<b7>>+<<c7>>+<<d7>>)) :  :={round(d7/sum(b7:d7),3)}} \\
% \hline
% \end{spreadtab}


\end{document}


\end{document}

